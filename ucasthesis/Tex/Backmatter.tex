\chapter{作者简介}

% \textbf{本科生无需此部分}。

\section*{作者简介:}
	姓名:林尽忠\quad 性别:男\quad 出生日期:1992.2.27\quad 籍贯:福建省古田县\\

	2012.9-2016.6	在杭州电子科技大学通信工程学院获得学士学位

	2016.9-2019.6 	在中国科学院大学计算机科学与技术学院攻读硕士学位

\section*{已发表(或正式接受)的学术论文:}

[1] Jinzhong Lin, Junbiao Pang, Li Su, Yugui Liu, Qingming Huang, "Accelerating Topic Detection on Web for a Large-Scale Data Set via Stochastic Poisson Deconvolution", in Proceedings of International Conference on Multimedia Modeling, 2019, pp. 590-602.

 \section*{获奖情况:}
 
 [1] 2019年被评为中国科学院“三好学生”

% (无专利时此项不必列出)

%\section*{参加的研究项目:}
%[1] 2015 年 8 月 - 2017 年 4 月,面向网络事件的跨平台异质媒体语义协同与挖掘,国
%家自然科学基金重点项目。课题编号: 61332016。

\chapter[致谢]{致\quad 谢}\chaptermark{致\quad 谢}% syntax: \chapter[目录]{标题}\chaptermark{页眉}
\thispagestyle{noheaderstyle}% 如果需要移除当前页的页眉
%\pagestyle{noheaderstyle}% 如果需要移除整章的页眉

时光荏苒,仿若白驹过隙。犹记得三年前独自一人来国科大面试,那情景,仿佛还发生在昨日,现在却到了要说再见的时候。这即将结束的三年北漂生活,同时也代表着学生生涯的结束。回望二十载的辛苦求学路,有近十载是独自异地求学,个中滋味,难以言表。在国科大的三年时光里,我不仅收获了很多知识,也得到了能力的提高和心理素质的锻炼。而这些,都离不开老师们的谆谆教诲,同学们的热情帮助以及家人朋友们的默默支持。在此,对你们致以衷心的感谢和祝福。

感谢我的父母。你们尽自己做大的努力给我提供良好的生活条件和求学环境,只愿我有更好的选择。从小到大,无论我做什么决定,你们总是无条件的支持我,鼓励我。相比学习成绩,更在乎我是否健康快乐。每每看到你们疲惫操劳的身影,我总是一阵心酸。只言片语无法表达我对你们的感谢和爱。祝愿二老身体健康,幸福快乐。

感谢黄庆明教授。在生活上对学生照顾有加,在科研上提供一流的设备,使我们能够心无旁骛地潜心科研。感谢您在面试阶段录取了我,给我打开一扇走入科研生活的大门,让我感受到学术的魅力。您严谨的科研态度、热情的关怀都使我铭记于心。在此也祝您身体健康,桃李满天下!感谢马丙鹏副教授,三年前,是您把懵懂的我招进中国科学院大学读研,从此走上科研之路,开始人生的新征程。感谢刘玉贵副教授在这三年对我的照顾,您严谨认真的治学态度和谦和豁达的人生态度着实令我钦佩。再次感谢马丙鹏老师和刘玉贵老师,祝您二位身体健康,事业顺利!

感谢庞俊彪副教授。感谢您这几年对我的指导和帮助。作为我的直接负责老师,您言传身教,真正做到了传道、授业、解惑。三年来,是您把我从一个科研门外汉,一步步带到门内。从课题的选择到算法研究,从实验开展到论文撰写,这其中的每一步都有您亲身参与,亲自指导。让我少走了很多弯路,同时也收获了很多。生活上,您对学生无微不至的体贴关怀;科研上,您对学生耐心有加,逐步指导。积极推动学生奋发向上,探索科研乐趣,不轻易放弃任何一个学生。您敏捷的思维逻辑、深刻的科研见解、深厚的学术功底、严谨的科研态度令我铭记于心。成为我不断学习,不断奋斗的目标榜样。感谢您带我度过这充实而难忘的三年时光。衷心祝愿庞老师阖家幸福,事业再上一层楼!

感谢实验室的许倩倩老师、王树徽老师、苏荔老师、李国荣老师、齐洪刚老师、李亮老师、张维刚老师和吴益灵师姐、杨智勇师兄、卓君宝师兄、吴哲师兄在学习中给我的帮助。感谢这三年唯一的室友和伙伴廖昌粟同学以及戚兆波、徐凯、辛永健、刘雪静、胡玲、郭双双等同学的陪伴,使得这三年的时光也有许多欢声笑语。在此祝愿所有的老师工作顺利,所有的同学们学业有成!

感谢未来的她,是你让我在迷茫困惑时有了坚持下来的动力!

最后,向百忙之中抽出宝贵时间评审本论文的专家和学者表示感谢!




\cleardoublepage[plain]% 让文档总是结束于偶数页,可根据需要设定页眉页脚样式,如 [noheaderstyle]

