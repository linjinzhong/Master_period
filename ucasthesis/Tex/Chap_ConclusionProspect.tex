\chapter{总结与展望}\label{chap:ConclusionProspect}

\section{本文工作总结}

本文致力于大规模网络数据的话题检测研究。基于多粒度网络话题的无监督排序算法,将话题检测问题转换为无监督的排序问题。本文提出两个算法来尽可能地解决大规模网络数据带来的效率问题。这两个算法分别针对网络话题的快速生成和网络话题的快速排序,并且在MCG-WEBV和YKS数据集上验证了我们算法效果和效率。同时我们还构造了一个较大规模的网络数据集来进一步验证我们的算法在大规模网络数据上的效率。

在网络话题的快速生成中,我们首先分析了网络话题在相似度空间上的统计模式,发现在相似度空间上的网络话题与L\'evyy Walks存在统计意义上的相似性,就是说一个话题内的所有网页之间的相似度服从一个未知参数的重尾分布。同时发现网络话题并不完全满足普遍所接受的看似合理的假设:话题内任意一个网页与话题内其他网页之间的类内相似度应该大于该网页与话题外其他网页之间的类间相似度。据此,我们提出了一种基于L\'evy Walks的话题生成算法:首先使用余弦距离构造网页之间的相似度图,并通过使用$k$近邻算法保留与网页最相似的$k$个网页,再使用阈值截断该相似度图,从而过滤大部分噪声网页;然后,我们通过站点熵率来评估每个网页成为话题中心网页的概率,并使用贪心算法来找到小规模的种子网页,初始化种子网页为种子话题;接着我们将网页带来的平均相似度与话题平均相似度的比值定义为网页与话题的相似度,然后基于该相似度将非噪声网页分配给多个种子话题。最后在分配时考虑当前话题的平均相似度,使用层级阈值进行截断,从而产生多粒度的网络话题。该算法简单高效,在MCG-WEBV和YKS数据集上,针对话题的召回率超过了当前最好的基于随机游走的非负矩阵分解算法,同时还极大的提高了效率。

在网络话题的快速排序中,我们首先回顾了PD算法,并分析其在处理大规模数据时的不足。然后由于基于EM算法来优化的PD算法不能够直接使用随机梯度下降来优化,所以我们找到了优化最小化原则,这是EM的泛化版本。接着我们使用随机优化最小化原则来优化PD算法,提出了SPD算法。该算法每轮迭代仅仅利用一小批相似度边来更新目标函数的上界代理函数,然后优化求解当前估计。由于每轮迭代仅仅利用一小批样本,使得SPD减少了物理内存的需求,同时提高了计算效率。在MCG-WEBV和YKS数据集上的实验验证SPD算法在话题检测方面的效果,同时在构造的人工数据集上的实验验证了SPD算法的效率。此外,基于多核系统,我们还实现了该算法的异步并行版本-AsySPD算法。



\section{未来研究展望}

随着社交媒体与移动网络的进一步发展与普及,人们会越来越多地投入到网络数据的创作和传播。因此,网络数据的规模只会越来越庞大,其中蕴含的有价值的信息也会越来越多。虽然本文提出的两个行之有效的算法比当前最好的网络话题检测算法效率更高,更具有可扩展性,但是离真正能处理超大规模网络数据,还有很长一段路,我们提出的算法只是在这条路上迈进了一步。因此,面向大规模网络数据的话题检测研究仍将是一项任重道远的任务。在未来的研究中,我们将着重从以下几点展开:
\begin{enumerate}
	\item[(1)] 在寻找种子网页的算法中,如果能找到一种更有效的评估网页作为话题中心网页的可能性,那么应该能进一步地提高话题的召回率;
	\item[(2)] 如果能找到一种更有效更合适的方法来度量网页跟话题的相似度,那么可能会进一步提高话题生成效率和效果;
	\item[(3)] 研究更高级的排序算法来提高LWTG算法所生成的话题的准确性;
	\item[(4)] 基于现实场景下的数据是信息流模式的情况,将在线学习用于网络话题检测是另一项有趣的研究。
\end{enumerate}
