%---------------------------------------------------------------------------%
%->> 封面信息及生成
%---------------------------------------------------------------------------%
%-
%-> 中文封面信息
%-
\confidential{}% 密级:只有涉密论文才填写
\schoollogo{scale=0.095}{ucas_logo}% 校徽
\title{面向大规模网络数据的话题检测研究}% 论文中文题目
\author{林尽忠}% 论文作者
\advisor{刘玉贵~副教授}% 指导教师:姓名 专业技术职务 工作单位
\advisors{中国科学院大学计算机科学与技术学院}% 指导老师附加信息 或 第二指导老师信息
\degree{硕士}% 学位:学士、硕士、博士
\degreetype{工程}% 学位类别:理学、工学、工程、医学等
\major{计算机技术}% 二级学科专业名称
\institute{中国科学院大学计算机科学与技术学院}% 院系名称
\date{2019~年~6~月}% 毕业日期:夏季为6月、冬季为12月
%-
%-> 英文封面信息
%-
\TITLE{Topic Detection Research for Large-Scale Web Data}% 论文英文题目
\AUTHOR{Jinzhong Lin}% 论文作者
\ADVISOR{Supervisor: Professor Yugui Liu}% 指导教师
\DEGREE{Master}% 学位:Bachelor, Master, Doctor。封面格式将根据英文学位名称自动切换,请确保拼写准确无误
\DEGREETYPE{Computer Technology}% 学位类别:Philosophy, Natural Science, Engineering, Economics, Agriculture 等
\MAJOR{Computer Science and Technology}% 二级学科专业名称
\INSTITUTE{School of Computer Science and Technology,\\ University of Chinese Academy of Sciences}% 院系名称
\DATE{June, 2019}% 毕业日期:夏季为June、冬季为December
%-
%-> 生成封面
%-
\maketitle% 生成中文封面
\MAKETITLE% 生成英文封面
%-
%-> 作者声明
%-
\makedeclaration% 生成声明页
%-
%-> 中文摘要
%-
\chapter*{摘\quad 要}\chaptermark{摘\quad 要}% 摘要标题
\setcounter{page}{1}% 开始页码
\pagenumbering{Roman}% 页码符号
随着信息技术和移动网络技术的快速发展,人们能够越来越方便地通过网络在社交媒体上获取信息和交换意见。因此,极大地促进了用户生成式内容的产生和传播。但是,海量的数据使得用户难以从中快速有效地提取当前热点话题以及感兴趣的话题。本文主要对当前网络话题检测在大规模网络数据上的可扩展性进行研究,在三个方面加以改进:网络话题生成,网络话题质量排序,并行化处理。

首先,本文研究了网络话题的生成任务。我们用相似度图表示网页之间的关系。由于网络中含有大量噪声网页,所以我们通过一定的阈值截断该相似度图并只保留最相关的一定个数的近邻网页间的相似度值。。。。

其次,本文研究了网络话题的质量排序。

最后,本文就质量排序的过程,进行了并行化的处理。

\keywords{网络话题检测,大规模网络数据,泊松去卷积算法}% 中文关键词
%-
%-> 英文摘要
%-
\chapter*{Abstract}\chaptermark{Abstract}% 摘要标题

Abstract:

\KEYWORDS{Topic Detection on Web, Large-Scale Web Data, Poisson Deconvolution}% 英文关键词
%---------------------------------------------------------------------------%
